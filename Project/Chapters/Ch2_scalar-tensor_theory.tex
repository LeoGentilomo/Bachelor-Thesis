
%%%%%%%%%%%%%%%%%%%%%%%%%%%%%%%%%%%%%%%%%%%%%%%%%%%%%%%%%%%%%%%%%%%%%%
%%%%%%%%%%%%%%%%%%%%%%%%%%% literature %%%%%%%%%%%%%%%%%%%%%%%%%%%%%%%
%%%%%%%%%%%%%%%%%%%%%%%%%%%%%%%%%%%%%%%%%%%%%%%%%%%%%%%%%%%%%%%%%%%%%%


\chapter{Scalar-Tensor theories} \label{Scalar-tensor theories}
\thispagestyle{empty}
Scalar-Tensor theories are generalizations of classical General Relativity with a scalar field that directly couples to the curvature scalar $R$.
There exist different kinds of Scalar-Tensor theories\footnote{For example, in the theory proposed by C. Brans and R. H. Dicke \citep{brans-dicke_theory} the scalar field introduces a spacetime dependence of Newton's gravitational constant G.}, here we present the action in the most general case. We notice that it is also possible to couple a scalar field such that it slightly affect the ordinary GR equations, then we indicate this frame as minimally coupled (MC) frame. In section \ref{Mapping MC and CC frames} we report a remarkable theorem to map the conformal and the MC frame. 


%-----------------------------------------------------------------------------%

\section{Coupling a scalar field to the action}
We are interested in new solutions that a scalar field could produce once incorporated into the theory. Therefore the most general way of introducing a scalar field $\phi$, is writing the action as 
\begin{equation}
    S[g_{\mu \nu}, \phi] = \int d^4x \sqrt{-g} \left[ \underbrace{f(\phi)R}_{\mathcal{L_R}}  - \underbrace{\frac{1}{2}h(\phi)\nabla^{\mu}\phi \nabla_{\mu}\phi - V(\phi)}_{\mathcal{L_{\phi}}} + \mathcal{L}_M(g_{\mu \nu}, \alpha_i)\right]
    \label{azione totale scalare tensoriale}
\end{equation}
where $\mathcal{L_{\phi}}$ is the pure scalar term, $f, h$ and $V$ are the fields defining the theory and $\mathcal{L}_M$, the matter lagrangian, depends on some fields $\alpha_i$, on the metric but not on $\phi$. Through a variation of this action it is possible to derive the equations of motion (a more precise procedure can be found in \citep{Carroll_2019}). 
Starting with 
\begin{equation*}
    g^{\mu \nu} \longrightarrow g^{\mu \nu} + \delta g^{\mu \nu}
\end{equation*}
leads to 
\begin{equation}
    G_{\mu \nu} = f^{-1} \left(\frac{1}{2}T_{\mu \nu}^M + \frac{1}{2}T_{\mu \nu}^{\phi} + \nabla_{\mu}\nabla_{\nu}f  - g_{\mu \nu} \Box f \right)
    \label{einstein scalare tensoriale}
\end{equation}
where $T_{\mu \nu}^M$ is obtained from the variation of $\mathcal{L}_M$ with respect to $g_{\mu \nu}$ and 
\begin{equation}
    T_{\mu \nu}^{\phi} = h \nabla_{\mu}\phi \nabla_{\nu}\phi - g_{\mu \nu}\left( \frac{1}{2}h\nabla^{\rho}\phi\nabla_{\rho}\phi + V(\phi)\right)
    \label{klein gordon scalare tensoriale}
\end{equation}

The next step is to produce a variation in the scalar field 
\begin{equation*}
    \phi \longrightarrow \phi + \delta \phi
\end{equation*}
that leads to 
\begin{equation}
    h \Box \phi + \frac{1}{2} \frac{d h}{d\phi} \nabla^{\mu}\phi \nabla_{\mu}\phi - \frac{d V}{d\phi} + R\frac{d f}{d\phi} = 0
\end{equation}
Finally a variation with respect to the fields $\alpha_i$ produces the associated matter equations.

Notice that if we take $f(\phi) = 1$, $h(\phi) = 1$ and $V(\phi)=\mathcal{L}_M=0$ in (\ref{azione totale scalare tensoriale}) we obtain the action of GR with a minimally coupled (MC) scalar field. 
%-----------------------------------------------------------------------------% 
\section{Coupling a conformal scalar field to the action}
\subsection{Conformal transformations}
Looking at (\ref{einstein scalare tensoriale}) and (\ref{klein gordon scalare tensoriale}), it's clear that working with a scalar field complicates the equations, so sometimes it could be useful to improve a conformal transformation to recover the appearance of classical GR. Let's begin by stating that given an n-dimensional manifold M with metric $\tilde{g}_{\alpha \beta}$ and a smooth function $\Omega$, we can define a new metric as $g_{\alpha \beta} = \Omega^2 \tilde{g}_{\alpha \beta}$, then we say that $\tilde{g}_{\alpha \beta}$ and $g_{\alpha \beta}$ are related by a \emph{conformal transformation}. Now we have a $\tilde{\nabla}$ operator associated to $\tilde{g}_{\alpha \beta}$ and $\nabla$ to $g_{\alpha \beta}$ that defines the relations between differential geometry quantities like the Riemann tensor. In particular it is possible to demonstrate (see appendix G of \citep{Carroll_2019} and D of \citep{wald2010general} for a complete derivation) that, for $n=4$ dimensions, the scalar curvature respect
\begin{equation}
    R = \Omega^{-2}\tilde{R} -6\tilde{g}^{\alpha \beta}\Omega^{-3}\tilde{\nabla}_{\alpha}\tilde{\nabla}_{\beta}\Omega
    \label{relazione R conforme e non}
\end{equation}
with $\nabla_{\alpha}\phi = \tilde{\nabla}_{\alpha}\phi = \partial_{\alpha}\phi$ for scalar fields. We can use this last relation, obviously with $g^{\alpha \beta} = \Omega ^{-2}\tilde{g}^{\alpha \beta}$, to recast the $\mathcal{L}_R$ term of the action (\ref{azione totale scalare tensoriale}) in the classical GR form. 
In particular taking (\ref{azione totale scalare tensoriale}) with $h = V = \mathcal{L}_M = 0$ in the $\tilde{g}_{\alpha \beta}$ frame
\begin{equation}
    S = \int d^4x\sqrt{-\tilde{g}}f(\phi)\tilde{R}. 
\end{equation}
Following this last action we define a conformal transformation such that $g_{\alpha \beta} = f(\phi) \tilde{g}_{\alpha \beta}$ (so $\Omega^2 = f(\phi)$) in order to replace $\tilde{R}$ from \ref{relazione R conforme e non} and $\sqrt{-\tilde{g}} = f^{-2}\sqrt{-g}$ obtaining
\begin{equation}
    S = \int d^4x \sqrt{-g}(R + 6g^{\alpha \beta} f^{-1/2}\nabla_{\alpha}\nabla_{\beta}f^{1/2}).
\end{equation}
The last term of the action can be integrated by part discarding the surface term, in particular 
\begin{equation}
    \int d^4x \sqrt{-g}f^{-1/2}\nabla^{\alpha}\nabla_{\alpha}f^{1/2}=\cancel{\nabla^{\alpha}(f^{-1/2}\nabla_{\alpha}f^{1/2})}-\int d^4x \sqrt{-g}\nabla^{\alpha}f^{-1/2}\nabla_{\alpha}f^{1/2}
\end{equation}
resulting in
\begin{equation}
    S=\int d^4x \sqrt{-g}\left[R + \frac{3}{2}f^{-2}\left(\frac{df}{d\phi}\right)^2\nabla^{\alpha}\phi \nabla_{\alpha}\phi\right]
\end{equation}
that represent the action of a Scalar-Tensor theory with the single term R, exactly  as in classical GR. For this similarity in appearance this frame is called \emph{Einstein frame} while the other frame, with $\tilde{R}f(\phi)$ in the action, is called \emph{Jordan} or \emph{string frame}.


\subsection{Conformal invariance}
Another interesting aspect of working with conformal transformation is \emph{conformal} or \emph{ scale invariance} of equations. An equation for a field $\phi$ is conformally invariant if there exist a number $s\in \mathbb{R}$, called the conformal weight of the field, such that $\phi$ is a solution with metric $\tilde{g}_{\alpha \beta}$ \emph{if and only if} $\psi = \Omega^s \phi$ is a solution with metric $g_{\alpha \beta} = \Omega^2\tilde{g}_{\alpha \beta}$. So when we say that we want to conformally couple a scalar field to our action, we mean that the equation of motion for the field derived from that action must be conformally invariant.

In general, in Jordan frame, we can ensure a conformal invariace for the scalar field equation taking the action (\ref{azione totale scalare tensoriale})
with (changing notation $\phi \rightarrow \psi$) : $f(\psi) = (1-\frac{k}{6}\psi^2)$, $h(\psi) = 2k$, $V(\psi)=\mathcal{L}_M = 0$ and $k = 8\pi$ (G=1) 
\begin{equation}
    S[g_{\mu \nu}, \psi] = \frac{1}{2k} \int d^4x \sqrt{-g} \left[ R - k\left( \nabla^{\mu}\psi \nabla_{\mu}\psi  + \frac{R}{6}\psi^2\right) \right]
    \label{Azione G.R conforme}
\end{equation}
 that produces the following equations of motion
\begin{equation}
    G_{\mu\nu} = k  \left[\partial_{\mu}\psi\partial_{\nu}\psi - \frac{1}{2}g_{\mu\nu}\partial^{\sigma}\psi\partial_{\sigma}\psi + \frac{1}{6}\left(g_{\mu\nu}\Box - \nabla_{\mu}\nabla_{\nu} + G_{\mu\nu}\right)\psi^2\right] 
    \label{Equazioni G.R conforme scariche}
\end{equation}
\begin{equation}
    \Box \psi - \frac{R}{6}\psi=0 
    \label{K.G. conforme}
\end{equation}
and this last equation, contrary to the classical one $\Box\psi = 0$, is conformally invariant (see \citep{wald2010general}). We indicate this frame as the conformally coupled frame (CC), to distinguish it from the MC frame where the equations of motion are not conformally invariant. 

\section{Mapping MC and CC frames} \label{Mapping MC and CC frames}
As seen before we can minimally couple a scalar field to the action of general relativity that slightly affect the ordinary equations, in particular we can write from \ref{azione totale scalare tensoriale} a theory with
\begin{equation}
    S[\hat{g}_{\mu \nu}, \phi] = \int d^4x \sqrt{-\hat{g}}\left[\hat{R}-k\nabla^{\alpha}\phi \nabla_{\alpha}\phi\right]
    \label{azione MC}
\end{equation}
that produces the following equations of motion
\begin{equation}
    G_{\mu \nu} = k \left(\nabla_{\mu}\phi \nabla_{\nu}\phi -\frac{1}{2}\hat{g}_{\mu \nu}\nabla^{\alpha}\phi \nabla_{\alpha}\phi \right)
    \label{EQ einstein MC}
\end{equation}
\begin{equation}
    \Box \phi = 0
    \label{K.G. MC}
\end{equation}
This equation is not conformally invariant, but what is extremely interesting to know is that Bekenstein found in \citep{mappaBekenstein} how to map the solutions of MC theory (\ref{azione MC}) into those of CC theory (\ref{Azione G.R conforme}).  The result is stated with the following theorem 
\theoremstyle{plain} % Stile del teorema
\newtheorem{theorem}{Theorem}
\begin{theorem}
if \( \hat{g}_{\mu \nu}, \Bar{F}_{\mu \nu}, \phi, \bar{\rho} \) and \(\bar{u}_v\) form a solution of Einstein's equations for a spacetime containing an ordinary scalar field \(\phi\), an electromagnetic field \(\Bar{F}_{\mu \nu}\) and radiation of density \(\bar{\rho}\) with 4-velocity \(\bar{u}_v\), then $g_{\mu \nu} = \Omega^2\hat{g}_{\mu \nu}$, $\psi=  \sqrt{\frac{6}{k}}\tanh{\sqrt{\frac{k}{6}}\phi}$, $F_{\mu \nu} = \Bar{F}_{\mu \nu}$, $\rho = \Omega^{-4}\bar{\rho}$, $u_v =\Omega \bar{u}_v$ is the corresponding solution for a conformal scalar field $\psi$, where $\Omega = \cosh{\sqrt{\frac{k}{6}}\phi}$.
\end{theorem}
We can summarize the result stating that given a solution ($\phi, \hat{g}_{\mu \nu}$) of the MC theory (\ref{azione MC}) we can find the corresponding solution  ($\psi, g_{\mu \nu}$) of the CC theory (\ref{Azione G.R conforme}) by applying the map
\begin{equation}
    \phi \longrightarrow \psi = \sqrt{\frac{6}{k}}\tanh{\left(\sqrt{\frac{k}{6}}\phi\right)},
    \notag
\end{equation}
\begin{equation}
    \hat{g}_{\mu \nu} \longrightarrow g_{\mu \nu} = \left[1-\frac{k}{6}\psi^2 \right]^{-1}\hat{g}_{\mu \nu}.
    \label{mappa bekenstein}
\end{equation}


