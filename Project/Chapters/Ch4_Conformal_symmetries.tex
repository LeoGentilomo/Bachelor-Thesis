
%%%%%%%%%%%%%%%%%%%%%%%%%%%%%%%%%%%%%%%%%%%%%%%%%%%%%%%%%%%%%%%%%%%%%%
%%%%%%%%%%%%%%%%%%%%% results and discussion %%%%%%%%%%%%%%%%%%%%%%%%%
%%%%%%%%%%%%%%%%%%%%%%%%%%%%%%%%%%%%%%%%%%%%%%%%%%%%%%%%%%%%%%%%%%%%%%


\chapter{The symmetry group} \label{Symmetries of CC Ernst's equations}
\thispagestyle{empty}
Here, we address one of the main topics of this thesis, the transformation symmetries of equations of motion (\ref{Ernst_CC_WLP})-(\ref{K.G._ernst_WLP}) of the conformal frame. In particular we take in consideration the continuous transformation group developed by the mathematician Sophus Lie
called Lie point symmetries \citep{stephani}-\citep{stephani_GR}. Given a system of differential equations, a Lie point symmetry is a change of variables that transform a solution of the system into a new solution. Actually with this method we are mapping an existing solution, called the seed solution, into a, possibly non trivial, new one. We will not discuss Lie point symmetry here; instead, we will directly apply the method as presented in \citep{Martelli}-\citep{corti} for our effective action (\ref{azione_effettiva_WLP_con_derivate_prime}). As said before, we are not completely blind on some of this symmetries. Indeed since the symmetry group has been applied in the MC frame \citep{embedding}, we expect to recover this MC symmetries mapped by Bekenstein's map (\ref{mappa bekenstein}) in the conformal frame.

\section{Symmetries of the effective action}
The starting point is considering the Lagrangian density of effective action (\ref{azione_effettiva_WLP_con_derivate_prime})
\begin{equation}
    \mathcal{L} = -\frac{3\alpha\nabla\chi^2}{\chi^2(\chi+3)}+ \frac{4}{3} \left(\frac{\nabla\alpha^2}{\alpha} -\alpha\frac{\nabla\mathcal{E}\nabla\mathcal{E}^*}{(\mathcal{E} + \mathcal{E}^*)^2} +\nabla\alpha\nabla\gamma + \alpha\frac{\nabla\gamma\nabla\rho}{\rho} \right) \notag
\end{equation}
from this we had to find the induced bilinear form. In particular we can make the substitution (see \citep{corti} for particular)
\begin{equation}
    \nabla \longrightarrow d \notag
\end{equation}
and rename the field $\mathcal{E}\rightarrow x +i y$ such that the Lagrangian above appear as a metric 
\begin{equation}
    ds^2=-\frac{3 \alpha}{\chi^2(\chi +3)}d\chi^2 + \frac{4}{3}\frac{d\alpha^2}{\alpha} - \frac{\alpha}{3x^2}(dx^2 + dy^2) +\frac{4}{3}d\alpha d\gamma +\frac{4}{3}\alpha\frac{d\rho d\gamma}{\rho}.
    \label{metrica da azione LWP}
\end{equation}
Notice that in order to find Lie point symmetries, we have to consider the coordinate $\rho$ as an unknown field, just like the others, in the action (\ref{azione_effettiva_WLP_con_derivate_prime}).
From the Killing vectors of the metric (\ref{metrica da azione LWP}) we can extract the transformations symmetries of equations of motion. 
\subsection{Killing vectors}
Let's start remembering that a Killing vector is a vector field that satisfy the Killing equation
\begin{equation}
    \nabla_{\mu}\xi_{\nu} + \nabla_{\nu}\xi_{\mu} = 0
    \label{killing equation}
\end{equation}
Killing vector are of particular interest in General Relativity as they identify the transformations under which the geometry is invariant. It is always possible to extract a conserved quantity with a killing vector field $\xi$ \footnote{In particular given $\xi$ and a geodesic $\gamma$ with tangent vector $U$, the quantity $U_{\mu}\xi^{\mu}$ is constant on $\gamma$.}. Here we are interested on Killing vectors because they represent the infinitesimal generators of the transformations and then, by integration, to the finite ones. The max number of Killing vectors on a N dimensional pseudo-Riemannian manifold is $\frac{1}{2}N(N+1)$, so for our metric (\ref{metrica da azione LWP}) there are at most 21 Killing vectors. In term of our metric, equation (\ref{killing equation}) becomes a system of 21 coupled partial differential equations reported below.
\begin{equation*}
    \begin{aligned}
        \begin{cases}
          \displaystyle \rho \xi_{\rho} - \alpha \xi_{\alpha}+ 8x\xi_x + 8x^2\partial_x\xi_{x} = 0 \\[0.5em]
          
          \displaystyle \frac{2}{x}\xi_y + \partial_y\xi_x + \partial_x\xi_y = 0 \\[0.5em]
          
          \displaystyle -\frac{\xi_x}{\alpha}+\partial_{\alpha}\xi_x + \partial_x\xi_{\alpha} =0 \\[0.5em]
          
          \displaystyle -8x\xi_x + \alpha\xi_{\alpha} -\rho \xi_{\rho} + 8x^2 \partial_y\xi_y =0\\[0.5em]
          
          \displaystyle -\frac{\xi_y}{\alpha} + \partial_{\alpha}\xi_y + \partial_y\xi_{\alpha} =0 \\[0.5em]

          \displaystyle - \frac{\xi_{\gamma}}{\alpha} + \frac{\xi_{\alpha}}{2} - \frac{\rho}{2\alpha}\xi_{\rho} + \partial_{\alpha}\xi_{\gamma} + \partial_{\gamma} \xi_{\alpha}=0\\[0.5em]

          \displaystyle \frac{\alpha}{2 \rho}\xi_{\alpha}-\frac{\xi_{\rho}}{2} + \partial_{\rho}\xi_{\gamma}+ \partial_{\gamma}\xi_{\rho}=0 \\[0.5em]

          \displaystyle \frac{\xi_{\alpha}}{\alpha} -\rho\frac{\xi_{\rho}}{\alpha^2} +2\partial_{\alpha}\xi_{\alpha}=0 \\[0.5em]

          \displaystyle -\frac{\xi_{\chi}}{\alpha} + \partial_{\chi}\xi_{\alpha}+ \partial_{\alpha}\xi_{\chi}=0 \\[0.5em]

          \displaystyle 9(\alpha\xi_{\alpha} - \rho\xi_{\rho}) + 12\chi(\chi+2)\xi_{\chi} + 8\chi^2(\chi+3)\partial_{\chi}\xi_{\chi}=0 \\[0.5em]

          \displaystyle -\frac{\xi_{\rho}}{\alpha}+ \partial_{\rho}\xi_\alpha + \partial_{\alpha}\xi_{\rho}=0 \\[0.5em]
          
        \end{cases}
        \qquad
        \begin{cases}
          \displaystyle \partial_{\gamma}\xi_x + \partial_x\xi_{\gamma}=0 \\[0.5em]

          \displaystyle \partial_{\chi}\xi_x + \partial_x\xi_{\chi}=0 \\[0.5em]

          \displaystyle \partial_{\rho}\xi_x + \partial_x\xi_{\rho}=0 \\[0.5em]

          \displaystyle \partial_{\gamma}\xi_y + \partial_y\xi_{\gamma}=0 \\[0.5em]

          \displaystyle \partial_{\chi}\xi_y + \partial_y\xi_{\chi}=0 \\[0.5em]

          \displaystyle \partial_{\rho}\xi_y + \partial_y\xi_{\rho}=0 \\[0.5em]

          \displaystyle \partial_{\chi}\xi_{\gamma} + \partial_{\gamma}\xi_{\chi}=0 \\[0.5em]

          \displaystyle \partial_{\chi}\xi_{\rho} + \partial_{\rho}\xi_{\chi}=0 \\[0.5em]

          \xi_{\rho}+\rho\partial_{\rho}\xi_{\rho}=0 \\[0.5em]

          \displaystyle \partial_{\gamma}\xi_{\gamma} =0 \\[0.5em]
        \end{cases}
    \end{aligned}
\end{equation*}
The solution of this system leads to the contravariant components (\ref{contravariant_components}) of the killing vector field (where we can rise indices with the inverse of (\ref{metrica da azione LWP}) ). Notice that there is nine different integration constant $c_{1-9}$ that will identify nine finite transformations.

\begin{equation}
    \begin{aligned}
        &\xi^x = 3x(2c_3 y-c_1) \\
        &\xi^y = -(3c_3(x^2-y^2) + 3c_1y+3c_2)\\
        &\xi^{\gamma} = \frac{3}{2}\operatorname{arctanh}\left[\sqrt{\frac{\chi +3}{3}}\right]c_6 + \frac{3}{2}(c_5-\gamma c_9)\\
        &\xi^{\alpha} =0 \\
        &\xi^{\chi} = -\frac{\chi \sqrt{\chi+3}}{6\sqrt{3}}(2\sqrt{3}c_4+3c_6(\log[\alpha] + \log[\rho]) + 3\gamma c_7)\\
        &\xi^{\rho} = \frac{3}{2}\rho\left(c_9(\log[\alpha]+\log[\rho])+\operatorname{arctanh}\left[\sqrt{\frac{\chi +3}{3}}\right]c_7 + c_8 \right)\\
        \label{contravariant_components}
    \end{aligned}
\end{equation}
Setting eight constant to zero and the last to one, nine different killing vectors or infinitesimal generators emerge 
\begin{equation}
    \begin{aligned}
        &\xi^{1} = -3(x\partial_x +y\partial_y)\\
        &\xi^{2} = -3\partial_y \\
        &\xi^{3} = 6xy\partial_x -3(x^2-y^2)\partial_y \\
        &\xi^{4} = -\frac{\chi\sqrt{\chi+3}}{3}\partial_{\chi}\\
        &\xi^{5} = \frac{3}{2}\partial_{\gamma}\\
        &\xi^{6} = \frac{3}{2}\operatorname{arctanh}\left[ \sqrt{\frac{\chi +3}{3}}\right]\partial_{\gamma} -\frac{\chi \sqrt{\chi+3}}{2\sqrt{3}}(\log[\alpha]+ \log[\rho])\partial_{\chi}\\
        &\xi^{7} = -\frac{\chi\sqrt{\chi+3}}{2\sqrt{3}}\gamma\partial_{\chi} +\frac{3}{2}\rho\operatorname{arctanh}\left[ \sqrt{\frac{\chi +3}{3}}\right]\partial_{\rho}\\
        &\xi^{8} = \frac{3}{2}\rho\partial_{\rho}\\
        &\xi^{9} = -\frac{3}{2}\gamma\partial_{\gamma} +\frac{3}{2}\rho(\log[\alpha]+\log[\rho])\partial_{\rho}\\
    \end{aligned}
    \label{killing vectors scarichi}
\end{equation}
From this vectors it is possible to generate the finite transformations.

\subsection{Finite transformations}\label{trasformazioni finite}
Following \citep{corti}, the components of infinitesimal generators $\xi^{1-9}$ represent the infinitesimal transformations of the field associated to the direction of that component. Let's take for example the generator $\xi^1$, the component in $x$ direction is $-3x$, this means that the transformed $x'(x,y,\epsilon)$ has the derivative with respect to the transformation parameter $\epsilon$ equals to $-3x'(x,y,\epsilon)$. Then the finite transformation is obtained by solving a system of differential equation
\begin{equation*}
    \begin{aligned}
        \begin{cases}
           \displaystyle \frac{dx'}{d\epsilon}  = -3x' \\[0.5em] 
           \displaystyle \frac{dy'}{d\epsilon}  = -3y' \\
        \end{cases}
        \rightarrow
        \begin{cases}
            \displaystyle x'=xe^{-3\epsilon} \\[0.5em] 
           \displaystyle y'=ye^{-3\epsilon} \\
        \end{cases}
    \end{aligned}
\end{equation*}
where we must impose the boundary condition $x'(x,y,0)=x$ and $y'(x,y,0)=y$, since if the parameter $\epsilon$ goes to zero there must be no transformation. The resulting $\mathcal{E}'$ is  $\mathcal{E}' = x'+iy'=e^{-3\epsilon}(x+iy)=e^{-3\epsilon}\mathcal{E}$. The same procedure for the other generators leads to nine different finite transformations reported below\footnote{We have omitted labels IV) and V) because, in general, this are attributed to a couple of transformation with the presence of an electromagnetic field.}
\begin{align}
  &I) & \mathcal{E} &\rightarrow \lambda\lambda^*\mathcal{E} & \alpha &\rightarrow \alpha & \chi \rightarrow \chi  &&  \gamma \rightarrow \gamma && \rho \rightarrow \rho \notag\\
  &II) &  \mathcal{E} &\rightarrow \mathcal{E}+ib & \alpha &\rightarrow \alpha & \chi \rightarrow \chi  &&  \gamma \rightarrow \gamma && \rho \rightarrow \rho \notag\\
  &III) &  \mathcal{E} &\rightarrow \frac{\mathcal{E}}{1+ic\mathcal{E}}& \alpha &\rightarrow \alpha & \chi \rightarrow \chi  &&  \gamma \rightarrow \gamma && \rho \rightarrow \rho \notag\\
  &VI) &  \mathcal{E} &\rightarrow \mathcal{E}& \alpha &\rightarrow \alpha & \chi \rightarrow \chi'  &&  \gamma \rightarrow \gamma && \rho \rightarrow \rho \notag\\ \notag
\end{align}
%\vspace{-30pt}
\begin{equation}
    \chi' = 3\left(\tanh^2\left[a \pm \operatorname{arctanh} \left[ \sqrt{\frac{\chi + 3}{3}}\right]\right] - 1\right) \notag
\end{equation}
\begin{align}
  &VII) & \mathcal{E} &\rightarrow \mathcal{E} & \alpha &\rightarrow \alpha & \chi \rightarrow \chi  &&  \gamma \rightarrow \gamma + d && \rho \rightarrow \rho \notag\\
   &VIII) & \mathcal{E} &\rightarrow \mathcal{E} & \alpha &\rightarrow \alpha & \chi \rightarrow \chi'  &&  \gamma \rightarrow \gamma' && \rho \rightarrow \rho \notag\\\notag
\end{align}
%\vspace{-50pt}
\begin{align}
        &\chi' = 3\left(\tanh^2\left[\frac{l}{2}(\log[\alpha]+ \log[\rho]) \pm \operatorname{arctanh} \left[ \sqrt{\frac{\chi + 3}{3}}\right]\right] - 1\right) \notag\\[0.7em]
        &\gamma' = \gamma + 3l\left(\frac{l}{4}(\log[\alpha]+ \log[\rho]) \pm \operatorname{arctanh} \left[ \sqrt{\frac{\chi + 3}{3}}\right]\right)\notag
\end{align}
%\vspace{-20pt}
\begin{align}
  &IX) & \mathcal{E} &\rightarrow \mathcal{E} & \alpha &\rightarrow \alpha & \chi \rightarrow \chi'  &&  \gamma \rightarrow \gamma && \rho \rightarrow \rho' \notag\\ \notag
\end{align}
%\vspace{-50pt}
\begin{align}
        &\chi' = 3\left(\tanh^2\left[\frac{l}{2}\gamma \pm \operatorname{arctanh} \left[ \sqrt{\frac{\chi + 3}{3}}\right]\right] - 1\right) \notag\\[0.7em]
        &\rho' = \rho\exp\left[3l\left(\frac{l}{4}\gamma \pm \operatorname{arctanh} \left[ \sqrt{\frac{\chi + 3}{3}}\right]\right)\notag \right] 
\end{align}
%\vspace{-10pt}
\begin{align}
        &X) & \mathcal{E} &\rightarrow \mathcal{E} & \alpha &\rightarrow \alpha & \chi \rightarrow \chi  &&  \gamma \rightarrow \gamma && \rho \rightarrow \nu\nu^*\rho \notag\\[0.7em]
        &XI) & \mathcal{E} &\rightarrow \mathcal{E} & \alpha &\rightarrow \alpha & \chi \rightarrow \chi  &&  \gamma \rightarrow \frac{\gamma }{p}&& \rho \rightarrow \rho^{p}\alpha^{p-1} \notag\\ \notag
\end{align}
where $a,b,c,d,l,p \in \mathbb{R}$ with $p>0$ and $\lambda, \nu \in \mathbb{C}$. Four of this transformations, the I, II, VII and X are gauge symmetries and can be reabsorbed by a change of coordinates, i.e. they do not lead to a nonequivalent physical solution. Transformation III is known as Ehlers transformation (present in classical GR as well as I and II) and it has the peculiarity of adding a NUT parameter to the solution. The NUT or gravomagnetic parameter can be interpreted as the dual of the mass exactly as the magnetic charge can be the dual of electric charge in electromagnetism \citep{enhanced}. The general Taub-NUT solution (see \citep{podolsky_exact_spacetimes}) does not present curvature singularity.
It is known \citep{embedding} that in the MC frame, the scalar field $\hat{\psi}$ presents the symmetry $\hat{\psi}\rightarrow\hat{\psi}+a$. If we apply Bekenstein's map to this symmetry, the result will be exactly the transformation VI (this is the symmetry which we expected to find). The remaining transformations VIII, IX and XI are new symmetries revealed thanks to Ernst's method directly applied in the conformal frame. We are not certain about how to implement symmetries involving the coordinate $\rho$. However, we can certainly observe that transformation IX, once understood how to apply it, may be used as a map between the classical theory and the conformal one. Indeed, it can introduce a non trivial scalar field from a solution of GR where $\psi=0$ ($\chi=4\pi\psi^2-3=-3$) thanks to the $\gamma$ term.

Once symmetries have been found in the uncharged case, it is straightforward to repeat the process for the theory (\ref{Azione G.R conforme elettromagnetica}) with electromagnetic field. Here we report the main results. The induced bilinear form of Lagrangian density $\mathcal{L}_{\mathcal{E}\boldsymbol{\Phi}}+\mathcal{L}_{\alpha\chi\gamma}$ of the action (\ref{azione_effettiva_WLP_con_derivate_prime_carica}) is 
\begin{equation}
        ds^2 = \frac{9 \alpha}{4\chi^2(\chi +3)}d\chi^2 - \frac{d\alpha^2}{\alpha} -d\alpha d\gamma -\alpha\frac{d\rho d\gamma}{\rho} +ds_{\mathcal{E}\boldsymbol{\Phi}}^2
        \label{metrica simmetrie carica}
\end{equation}
where if we define $\mathcal{E}$ and $\boldsymbol{\Phi}$ as $\mathcal{E} = x+iy$, $\boldsymbol{\Phi} = z+iw$ the last term becomes   
\begin{equation}
    ds_{\mathcal{E}\boldsymbol{\Phi}}^2 = \alpha\frac{dx^2+dy^2-12dx(zdz+wdw)-12y(zdw-wdz)+12x(dz^2+dw^2)}{4(x-3(z^2+w^2))^2}. \notag
\end{equation}
From metric (\ref{metrica simmetrie carica}) we can derive the following killing vectors 
\begin{equation}
    \begin{aligned}
        &\xi^{1} = 2x\partial_x + 2y\partial_y + z\partial_z + w\partial_w \\
        &\xi^{3} = 2xy\partial_x -(x^2-y^2)\partial_y + (wx+yz)\partial_z + (wy-xz)\partial_w \\
        &\xi^{10} = 2(wy-zx)\partial_x - 2(wx+yz)\partial_y + (2w^2-2z^2+\frac{x}{3})\partial_z - (4wz-\frac{y}{3})\partial_w \\
        &\xi^{11} = 2(wx+zy)\partial_x + 2(wy-zx)\partial_y + (4wz+\frac{y}{3})\partial_z + (2w^2-2z^2-\frac{x}{3})\partial_w \\
        &\xi^{12} = 2(z\partial_x + w\partial_y)+\frac{1}{3}\partial_z \\
        &\xi^{13} = 2(-w\partial_x + z\partial_y)-\frac{1}{3}\partial_w \\
        &\xi^{14} = w\partial_z -z\partial_w \\
    \end{aligned}
\end{equation}
the remaining ones $\xi^{2}$, $\xi^{4-9}$ are the same (up to a multiplicative coefficient) of the uncharged case (\ref{killing vectors scarichi}). From them, it is clear that transformations II, VI, VII, VIII, IX, X and XI are unchanged with the trivial addition  $\boldsymbol{\Phi}\rightarrow\boldsymbol{\Phi}$. The vectors $\xi^{10}$ and $\xi^{11}$ combine together to produce transformation V and the same for $\xi^{12}$ and $\xi^{13}$ to produce IV. $\xi^1$, $\xi^2$ and $\xi^{14}$ are modifications of transformations I and III as below 
\begin{align}
  &I) & \mathcal{E} &\rightarrow \lambda\lambda^*\mathcal{E} & \boldsymbol{\Phi} &\rightarrow \lambda\boldsymbol{\Phi} & \alpha &\rightarrow \alpha  & \chi \rightarrow \chi  &&  \gamma \rightarrow \gamma && \rho \rightarrow \rho \notag\\
  &III) &  \mathcal{E} &\rightarrow \frac{\mathcal{E}}{1+ic\mathcal{E}} & \boldsymbol{\Phi} &\rightarrow \frac{\boldsymbol{\Phi}}{1+ic\mathcal{E}} & \alpha &\rightarrow \alpha & \chi \rightarrow \chi  &&  \gamma \rightarrow \gamma && \rho \rightarrow \rho \notag\\
  &IV) & \mathcal{E} &\rightarrow \mathcal{E}-6\beta^*\boldsymbol{\Phi}+3\beta\beta^* & \boldsymbol{\Phi} &\rightarrow \boldsymbol{\Phi} + \beta & \alpha &\rightarrow \alpha & \chi \rightarrow \chi  &&  \gamma \rightarrow \gamma && \rho \rightarrow \rho \notag\\
  &V) &  \mathcal{E} &\rightarrow \frac{\mathcal{E}}{1+3\Tilde{\alpha}\Tilde{\alpha}^*\mathcal{E}-6\Tilde{\alpha}^*\boldsymbol{\Phi}} & \boldsymbol{\Phi} &\rightarrow \frac{\boldsymbol{\Phi}-\Tilde{\alpha}\mathcal{E}}{1+3\Tilde{\alpha}\Tilde{\alpha}^*\mathcal{E}-6\Tilde{\alpha}^*\boldsymbol{\Phi}} & \alpha &\rightarrow \alpha & \chi \rightarrow \chi  &&  \gamma \rightarrow \gamma && \rho \rightarrow \rho \notag\\ 
  \notag
\end{align}
with $\lambda,\beta, \Tilde{\alpha} \in \mathbb{C}$ and $c \in \mathbb{R}$. Again, we have that transformations I and IV are gauge symmetries while the III is the electromagnetic version of Ehlers transformation. 
Finally the V is called Harrison transformation and, in classical GR can be used to add electric charge to vacuum solutions.