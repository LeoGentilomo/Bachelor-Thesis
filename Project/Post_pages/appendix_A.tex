
%%%%%%%%%%%%%%%%%%%%%%%%%%%%%%%%%%%%%%%%%%%%%%%%%%%%%%%%%%%%%%%%%%%%%%
%%%%%%%%%%%%%%%%%%%%%%%%%%% appendix %%%%%%%%%%%%%%%%%%%%%%%%%%%%%%%%%
%%%%%%%%%%%%%%%%%%%%%%%%%%%%%%%%%%%%%%%%%%%%%%%%%%%%%%%%%%%%%%%%%%%%%%
\chapter{Ernst's technique in CC frame with a different ansatz} \label{appendice A}
Here we repeat the procedure to derive Ernst's method, and the correlated effective action, starting from a completely different anstaz with respect to the conformal LWP. In this case as well, the new metric must preserve the properties of stationarity and axial symmetry of the spacetime.
\section{The ansatz}
This ansatz is the same used in \citep{charging} to charging Ernst’s solution generating technique with cosmological constant and in \citep{Delice_stationary} to produce stationary and axisymmetric solution of Brans-Dicke theory. 
The ansatz is
\begin{equation}
        ds^2 = -\alpha e^{\Omega/2} (dt + \omega d\varphi)^2 + \alpha e^{-\Omega/2}d\varphi^2 + \frac{e^{2\nu}}{\sqrt{\alpha}}(d\rho^2+dz^2) 
        \label{ansatz_charging}
\end{equation}
with, thanks to the symmetry conditions, all function depending only on $\rho, z$.
\subsection{Deriving Ernst's equation}
It is possible to describe the five unknown fields, $\alpha,\Omega,\nu, \omega, \psi$, by four Einstein equations plus a Klein-Gordon equation in term of the ansatz (\ref{ansatz_charging})
\begin{equation}
        \nabla\cdot \left(\alpha \chi e^{\Omega}\nabla \omega \right) =0 \tag{$EE^{\varphi}_t$}
\end{equation}
\begin{equation}
    \nabla\cdot(e^{\Omega}\alpha\chi\omega\nabla\omega) + \frac{1}{2} \nabla\cdot(\alpha\chi\nabla\Omega) = 0 \tag{$EE^t_t- EE^{\varphi}_{\varphi}$}
\end{equation}
\begin{equation}
    \nabla^2(\alpha\chi)=0 \tag{$EE^{\rho}_{\rho}+EE^{z}_{z}$}
\end{equation}
\begin{equation}
    \frac{1}{2}\alpha\chi e^{\Omega}\nabla \omega^2 - \frac{1}{8}\alpha\chi\nabla\Omega^2 -2 \alpha\chi\nabla^2\nu + \frac{3}{2}\alpha\frac{\nabla\chi^2}{\chi + 3} - \frac{3}{2}\alpha \nabla^2\chi=0 \tag{$EE^t_t+ EE^{\varphi}_{\varphi}$}
\end{equation}
\begin{equation}
    6\nabla\cdot(\alpha\nabla\psi) + \frac{3}{2}\psi\nabla^2\alpha = \frac{1}{2}\alpha\psi e^{\Omega}\nabla \omega^2 -2\alpha\psi\nabla^2\nu-\frac{1}{8}\alpha\psi\nabla\Omega^2 \tag{K.G.}
\end{equation}
where has been defined $\chi \equiv 4\pi\psi^2 -3$ and we are using $\nabla^2 \equiv \partial_{\rho}^2 + \partial_z^2$. As done before we want to rewrite this five equation with Ernst potentials and then find the associated effective action. Again the equation $EE^{\varphi}_t$ is an integrability condition, in particular can be used to defined a new field $h$ such that
\begin{equation}
    \hat{e}_{\varphi} \times \nabla h = \alpha\chi e^{\Omega}\nabla\omega
    \label{definizione_h}
\end{equation}
that satisfies  
\begin{equation}
    \nabla\cdot\left[ (\alpha\chi e^{\Omega})^{-1}\nabla h \right] = 0
    \label{divergenza_h_charging}
\end{equation}
equivalent to $EE^{\varphi}_t$, and taking the square of (\ref{definizione_h})
\begin{equation}
    \nabla\omega^2 = (\alpha\chi e^{\Omega})^{-2}\nabla h^2.
    \label{h_due}
\end{equation}
With this last equation and defining $f \equiv \alpha\chi e^{\Omega /2}$ it is possible to rewrite $EE^t_t- EE^{\varphi}_{\varphi}$ as 
\begin{equation}
        f (\alpha\chi)^{-1}\nabla\cdot(\alpha\chi\nabla f) = \nabla f\nabla f - \nabla h^2
        \label{ernst_charging_con_f_esplicito}
\end{equation}
where has been neglected the $\nabla^2(\alpha\chi)$ term thanks to $EE^{\rho}_{\rho}+EE^{z}_{z}$. Now it's natural to define a complex potential $\mathcal{E} \equiv f + ih$, named Ernst potential, so that equations (\ref{ernst_charging_con_f_esplicito}) and (\ref{divergenza_h_charging}) can be viewed as real and imaginary part of a single complex equation, named Ernst equation 
\begin{equation}
    \mathrm{Re}\mathcal{E}\Tilde{\alpha}^{-1}\nabla(\Tilde{\alpha}\nabla\mathcal{E})=\nabla\mathcal{E}\nabla\mathcal{E}
    \label{Ernst_CC_charging}
\end{equation}
where $\Tilde{\alpha}\equiv \alpha\chi$. 
Once Ernst potential has been defined and making the substitutions $\alpha\chi \rightarrow \Tilde{\alpha}$, $\nabla\omega^2 \rightarrow \nabla h^2$, $\Omega\rightarrow 2\log(f/\Tilde{\alpha})$, the $EE^{\rho}_{\rho}+EE^{z}_{z}$, $EE^t_t+ EE^{\varphi}_{\varphi}$ and K.G. equations takes the form
\begin{equation}
    2\Tilde{\alpha}\nabla^2\nu = \frac{3\Tilde{\alpha}}{2\chi}\left(\frac{\nabla\chi^2}{\chi +3} - \nabla^2\chi \right) - \frac{\nabla\Tilde{\alpha}^2}{2\Tilde{\alpha}} - \Tilde{\alpha}\frac{\nabla\mathcal{E}^2 + {\nabla\mathcal{E}^*}^2}{(\mathcal{E} + \mathcal{E}^*)^2} + \frac{\nabla\mathcal{E} + \nabla\mathcal{E}^*}{\mathcal{E} + \mathcal{E}^*}\nabla\Tilde{\alpha} 
    \label{equazione_per_nu_ernst_charging}
\end{equation}
\begin{equation}
    \nabla^2\Tilde{\alpha}=0 
    \label{laplaciano_nullo}
\end{equation}
\begin{equation}
    6\nabla(\Tilde{\alpha}\nabla\chi) - \frac{9\Tilde{\alpha}(\chi +2)}{\chi(\chi + 3)}\nabla\chi^2 = 0 .
    \label{K.G._ernst_charging}
\end{equation}
\vspace{-3em}
\subsection{The effective action}
Finally is straightforward to demonstrate that the four equation of motion (\ref{Ernst_CC_charging})-(\ref{K.G._ernst_charging}) can be derived from the following effective action
\begin{align}
        S = \int d\rho dz d\varphi \frac{3 \Tilde{\alpha}}{\chi^2(\chi+3)} \nabla\chi^2 &- \chi^{-1} \nabla\Tilde{\alpha}\nabla\chi \notag + \\ &\frac{1}{3}\left[4\Tilde{\alpha}\frac{\nabla\mathcal{E} \nabla\mathcal{E}^*}{(\mathcal{E}+\mathcal{E}^*)^2} + \frac{\nabla\Tilde{\alpha}^2}{\Tilde{\alpha}} -2\frac{\nabla\mathcal{E}+\nabla\mathcal{E}^*}{\mathcal{E}+\mathcal{E}^*}\nabla\Tilde{\alpha} -4\nabla\Tilde{\alpha}\nabla\nu \right]
\end{align}
more precisely 
\begin{align}
    &\nabla \frac{\delta\mathcal{L}}{\delta\nabla\mathcal{E}} - \frac{\delta\mathcal{L}}{\delta\mathcal{E}} =0 \longrightarrow (\ref{Ernst_CC_charging}) \notag \\
    &\nabla \frac{\delta\mathcal{L}}{\delta\nabla\chi} - \frac{\delta\mathcal{L}}{\delta\chi} =0 \longrightarrow (\ref{K.G._ernst_charging}) \notag \\
    &\nabla \frac{\delta\mathcal{L}}{\delta\nabla\nu} - \frac{\delta\mathcal{L}}{\delta\nu} =0 \longrightarrow (\ref{laplaciano_nullo}) \notag \\
    &\nabla \frac{\delta\mathcal{L}}{\delta\nabla\Tilde{\alpha}} - \frac{\delta\mathcal{L}}{\delta\Tilde{\alpha}} =0 \longrightarrow (\ref{equazione_per_nu_ernst_charging}) \notag.
\end{align}



