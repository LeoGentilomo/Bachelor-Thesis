%%%%%%%%%%%%%%%%%%%%%%%%%%%%%%%%%%%%%%%%%%%%%%%%%%%%%%%%%%%%%%%%%%%%%%
%%%%%%%%%%%%%%%%%%% required packages and settings %%%%%%%%%%%%%%%%%%%
%%%%%%%%%%%%%%%%%%%%%%%%%%%%%%%%%%%%%%%%%%%%%%%%%%%%%%%%%%%%%%%%%%%%%%

%----------------------------------------------------------------
%				 		page and paragraph setup 
%----------------------------------------------------------------
\parindent=0cm 				% pragraph indentation = 0
\usepackage{setspace}		% for line spacing
\linespread{1.25}			% you can set line spacing to 1.5 also

%\usepackage{times} 		% for times roman font

\sloppy 					% for hyphenless justification
\hyphenation{} 				% use hyphenation of tolerance parameters,
\hyphenpenalty=10000
\exhyphenpenalty=10000

%----------------------------------------------------------------
%				 		header and footer setup 
%----------------------------------------------------------------

\usepackage{fancyhdr}		% required package

\pagestyle{fancy}
\fancyhf{}
\fancyhead[LE,RO]{\bfseries\thepage}
\fancyhead[RE,LO]{\itshape\nouppercase{\leftmark}}

%----------------------------------------------------------------
%				 		chapter title formatting 
%----------------------------------------------------------------

\usepackage{titlesec}		% to format titles and headings
\usepackage[titletoc]{appendix}

\titleformat{\chapter}		% it has four arguments
[display]					%
{\bfseries\Huge}			% format
{Chapter \ \thechapter}		% label
{20 pt}						% verticle spacing
{}							% before chapter number
[{\titlerule[1.5pt]}]		% a 1.5pt thick line after chapter title

\titlespacing*{\chapter}{0pt}{-30pt}{40pt}

% Formattazione specifica per le appendici
\newcommand{\appendixchapterformat}{%
    \titleformat{\chapter} 
    [display]
    {\bfseries\Huge}
    {Appendix \ \thechapter}   % cambia "Chapter" in "Appendix"
    {20 pt}
    {}
    [{\titlerule[1.5pt]}]
}
%--------------------- sections heading formatting --------------
% you can use following commands to format section and subsection heading

%\titleformat{\section}		% it has four arguments
%{}							% for formatting (size, font,...)
%{\thesection}				% numbering
%{}							% space between number and title
%{}							% any object between number and title

%------------------- subsections heading formatting -------------

%\titleformat{\subsection}	% it has four arguments
%{\bfseries\normalsize}		% for formatting (size, font,...)
%{}							% numbering
%{0 cm}						% space between number and title
%{} 						% any object between number and title

%----------------------- mathematics related -------------------
\usepackage{amsmath, amsthm, amssymb}		% for mathematic symbols
\usepackage[utf8x]{inputenc}				% for characters
\usepackage{bm}								% for bold font with mathematics
%ROBA CHE HO AGGIUNTO IO:
\usepackage{amsfonts}
\usepackage{tikz-cd}
\usepackage{chngcntr}
\usepackage[utf8x]{inputenc}
\usepackage{enumerate}
\usepackage{mathrsfs} % For script fonts
\usepackage{cancel}

%-------------------- links and cross-ref related ---------------

\usepackage[colorlinks=true, breaklinks, linkcolor=black, citecolor=blue]{hyperref} % for creating hyperlinks in cross-references
\usepackage{url}							% to insert url in the document
\usepackage[numbers]{natbib}				% to cite references

%----------------------------------------------------------------
%							for list of symbols
%----------------------------------------------------------------

\usepackage[intoc]{nomencl}					% for nomenclature
\makenomenclature
\renewcommand{\nomname}{List of Symbols}	% change title to list of symbols

%----------------------------------------------------------------

\usepackage[nottoc]{tocbibind} % needed for displaying bibliography and other in the table of contents


%----------------------------------------------------------------
%				 			for figures
%----------------------------------------------------------------

\usepackage{graphicx} 		% needed for \includegraphics
\usepackage{caption}		% for subfigures
\usepackage{subcaption}		% for subfigures

%----------------------------------------------------------------


\usepackage{multicol}		% to merge columns of table
\usepackage{multirow}		% to merge rows of table
\usepackage{longtable} 		% for long tables over pages

\usepackage{enumerate} 		% needed for some options in enumerate
\usepackage{cite}			% for multiple reference citation

\usepackage{todonotes} 		% needed for todos
\usepackage{makeidx} 		% needed for creating an index

%----------------------------------------------------------------
%				 		make your own new command
%----------------------------------------------------------------

\newcommand{\tb}{\textcolor{blue}}	% for blue colored text
\newcommand{\dc}{$^\circ$C}			% for degree Celcius symbol

% below, a new command \pde is created to write partial diffential eqn, which takes two arguments. An application of this command can be found in Ch 2 
\newcommand{\pde}[2]{\dfrac{\partial^2 #1}{\partial #2}}


