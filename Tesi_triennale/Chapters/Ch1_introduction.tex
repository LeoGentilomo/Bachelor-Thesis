
%%%%%%%%%%%%%%%%%%%%%%%%%%%%%%%%%%%%%%%%%%%%%%%%%%%%%%%%%%%%%%%%%%%%%%
%%%%%%%%%%%%%%%%%%%%%%%% introduction %%%%%%%%%%%%%%%%%%%%%%%%%%%%%
%%%%%%%%%%%%%%%%%%%%%%%%%%%%%%%%%%%%%%%%%%%%%%%%%%%%%%%%%%%%%%%%%%%%%%


\chapter{Introduction}
\thispagestyle{empty}

\section{Introduction}
General Relativity (GR), as developed by Albert Einstein in 1915, is the universally accepted theory of gravity thanks to the variety of experimental tests that confirm its prediction. The theory owes part of its success to the remarkable interpretation of the field that mediates gravitational interaction. Drawing on new mathematical tools such as Riemannian geometry, Einstein reinterpreted the motion of masses in space not as governed by mutual forces, but rather as motion along straight paths in a space curved by the presence of mass - 
the gravitational field itself.
This wonderful interpretation of the gravitational dynamics is synthesized by Einstein's field equations\footnote{Later modified with the addition of the cosmological constant term $\Lambda g_{\mu \nu}$}.
\begin{equation}
    R_{\mu\nu}-\frac{1}{2}Rg_{\mu \nu} = \frac{8\pi G}{c^4}T_{\mu \nu}.
    \label{equazioni di campo GR}
\end{equation}
Nevertheless, many cosmological and theoretical phenomena, such as dark energy and cosmological inflation, remain unexplained. It thus appears necessary to extend classical General Relativity to account for these phenomena. One of the most straightforward approaches to achieving this is through the use of a Scalar-Tensor theory.
Scalar-Tensor theories are those generalizations of classical GR that describe gravity using the classical metric tensor $g_{\mu \nu}$ and a scalar field $\phi$ directly coupled with the curvature scalar. The addition of a scalar field introduces new solutions and symmetries to the equations of classical GR, potentially leading to new gravitational phenomena.
\vspace{0.5 pt} 

Another challenge of the classical theory lies in solving the field equations (\ref{equazioni di campo GR}), as they constitute a system of 10 second-order nonlinear partial differential equations. Since the birth of the theory, various techniques for generating solutions to Einstein's equations have emerged, including the method developed by F. J. Ernst \citep{ernst1} for axisymmetric and stationary spacetimes. In particular, Ernst rearranged equations (\ref{equazioni di campo GR}) defining a complex potential $\mathcal{E}$ that brings out new symmetries into the system. These new symmetries are redefinitions or Gauge transformations of the Ernst's potential $\mathcal{E}$ except for a couple of them that lead to an inequivalent spacetime. Effectively, Ernst's method is a technique that allows one to generate a solution, from a ``seed'' one, that sometimes posses a different physical interpretation. 
\vspace{0.5 pt} 

In this thesis we export Ernst's method to a Scalar-Tensor theory where the new field $\phi$ is conformally coupled to the action (so that the equation derived from the field posses conformal invariance) and we explore the possible emergence of new symmetries with respect to classical GR. We are particularly interested in conformal invariance because Maxwell's equations, which we will couple to the Scalar-Tensor theory in Section \ref{tecnica con elettromagnetismo}, exhibit this property (see the appendix of \citep{wald2010general}). So the additional scalar field that we introduce is intended to be similar to the electromagnetic field, whose coupling with General Relativity is well-established. Furthermore, there is a historical significance in considering the conformal frame: it was within this framework that Bekenstein \citep{BBMB_sol} discovered the first counterexample to the ``No-hair'' theorem\footnote{The famous physicist John Wheeler coined this expression stating that ``Black holes have no hair'' \citep{Pagine_gialle}} (commonly referred to as the BBMB solution).
In chapter \ref{Scalar-tensor theories} we show how the action of a Scalar-Tensor theory appears, we present the case of interest for us (where the scalar field is conformally coupled to the action) and we introduce the notions of conformal transformation and conformal invariance. In chapter \ref{solution generating technique} we recast Ernst's method and Ernst's equations for our theory following step by step the original derivation in the vacuum case and in the electrovacuum case, so that also Maxwell's electromagnetism is included into the theory. Then in chapter \ref{Symmetries of CC Ernst's equations} we find the correlated Lie point symmetries and investigate how those present in the classical theory are modified or whether the coupling in the conformal frame reveals new ones. 
Finally in chapter \ref{applications}-\ref{new solutions} we first test the symmetries of Ernst's equations building some existing solution starting from the BBMB solution as seed and then we produce some new solutions of the conformal theory. In general we will use the signature $(-,+,+,+)$ and $G=c=\mu_0=1$.
