
%%%%%%%%%%%%%%%%%%%%%%%%%%%%%%%%%%%%%%%%%%%%%%%%%%%%%%%%%%%%%%%%%%%%%%
%%%%%%%%%%%%%%%%%%%%%%%%%%% conclusion %%%%%%%%%%%%%%%%%%%%%%%%%%%%%%%
%%%%%%%%%%%%%%%%%%%%%%%%%%%%%%%%%%%%%%%%%%%%%%%%%%%%%%%%%%%%%%%%%%%%%%


\chapter{Testing the solutions generating technique on BBMB black hole} \label{applications}
\thispagestyle{empty}
Before using the new symmetry transformations of chapter \ref{Symmetries of CC Ernst's equations}, we can try to build some known solution of the conformal theory, to verify that the symmetries actually work. In classical GR it is known that the transformation III, or Ehlers transformation, can incorporate the Taub-NUT parameter into a considered solution (in \citep{Martelli} it is shown how to apply it to Schwarzschild). At the same time in \citep{Caldarelli_Charmousis} has been found how to include this parameter in the BBMB solution of the conformal theory. So in section \ref{sezione bbmb+nut} we test the solutions generating technique by applying the transformation III at BBMB to obtain BBMB + NUT of \citep{Caldarelli_Charmousis}. 
Then it is also known that the Harrison transformation, when applied with a real parameter $\Tilde{\alpha} \in \mathbb{R}$, introduces an electric charge into an initial vacuum solution. In section \ref{sezione bbmb + harrison}, we demonstrate that our transformation V from Chapter \ref{Symmetries of CC Ernst's equations} has the same capability by applying it to BBMB (where $\Bar{A} = 0$). In appendix \ref{appendice C}, through the transformation III, we embed the BBMB black hole in a swirling background, that is the second known property of the Ehelrs transformation. 

\section{Adding NUT parameter to BBMB black hole} \label{sezione bbmb+nut}
The BBMB solution was the first counter example of the ``No-hair'' theorem and at the same time one of the first solution of equations (\ref{Equazioni G.R conforme scariche})-(\ref{K.G. conforme}) of the conformal theory. It is a spherically symmetric and static solution proposed by Bekenstein in \citep{mappaBekenstein} that reads in spherical coord. as 
\begin{equation}
    ds^2 = -\left(1-\frac{m}{r}\right)^2dt^2 + \left(1-\frac{m}{r}\right)^{-2}dr^2 + r^2(d\theta^2 + \sin^2\theta d\varphi^2)
    \label{BBMB metrics}
\end{equation}
with 
\begin{equation}
    \psi = \sqrt{\frac{6}{k}}\frac{m}{r-m}. \notag
\end{equation}
In \citep{C-metric} are listed some critical aspects of this solution. As mentioned, following \citep{Caldarelli_Charmousis}, Taub-NUT parameter can be included in solution (\ref{BBMB metrics}) to produce 
\begin{equation}
    ds^2 = - \frac{(r-\mu)^2}{r^2 + n^2}(dt+2n\cos\theta d\varphi)^2 + \frac{r^2 + n^2}{(r-\mu)^2}dr^2 + (r^2+n^2)(d\theta^2 + \sin^2\theta d\varphi^2),
    \label{BBMB_+_NUT_metric}
\end{equation}
\begin{equation}
    \psi=\sqrt{\frac{6}{k}}\frac{\sqrt{\mu^2+n^2}}{r-\mu} \notag
\end{equation}
where $\mu = \sqrt{m^2-n^2}$ and $n$ is the NUT parameter. Let's specify that this solution  is obtained from a symmetry of Ernst's equation in prolate spherical coordinates that correspond (has shown in \citep{enhanced}) to the composition $III \circ II$. Summarizing, the purpose is writing solution (\ref{BBMB_+_NUT_metric}) from the application of $III \circ II$ on (\ref{BBMB metrics}).

We can start by defining the transformation from cylindrical to spherical coordinates $\{\rho,\varphi,z\} \rightarrow \{r,\theta,\varphi\}$
\begin{equation*}
    \begin{aligned}
        \begin{cases}
           \rho  = (r-m)\sin\theta \\
           z = (r-m)\cos\theta \\
           \varphi = \varphi \\
        \end{cases}
    \end{aligned}
    \tag{$c\rightarrow s$}
    \label{cambio cilindriche sferiche}
\end{equation*}
with 
\begin{equation}
    d\rho^2 + dz^2 = dr^2 + (r-m)^2d\theta^2
\end{equation}
so that the conformal LWP ansatz becomes 
\begin{equation}
    ds^2 = \Omega\left[-f_0 (dt - \omega_0 d\varphi)^2 + f_0^{-1}\left(e^{2\gamma}(dr^2 + (r-m)^2d\theta^2) + (r-m)^2\sin^2\theta d\varphi^2\right)\right].
    \label{WLP metric sferiche}
\end{equation}
Now it's easy to check that with the substitutions below, the metric (\ref{WLP metric sferiche}) corresponds exactly to BBMB solution (\ref{BBMB metrics}).
\begin{equation*}
    \begin{aligned}
        \begin{cases}
           \displaystyle f_0  = \frac{(r-m)^2}{r^2} \\[0.5em] 
           \displaystyle \chi = 4\pi\psi^2-3 = 3r\frac{2m-r}{(r-m)^2} \\
        \end{cases}
        \begin{cases}
            \omega_0 = 0 \\
            \Omega = 1 \\
            \gamma = 0 \\
        \end{cases}
    \end{aligned}
\end{equation*}
Remembering that we can set $h_0=0$ in the static case $\omega_0=0$, let's apply the transformation $II$ to $\mathcal{E}_0 = \Tilde{f}_0 = \Omega\chi f_0 = -3\left(1-\frac{2m}{r}\right)$, in particular
\begin{equation}
    \mathcal{E}_0 \longrightarrow \mathcal{E} = \mathcal{E}_0 +ib = -3\left(1-\frac{2m}{r}\right) + ib
\end{equation}
followed by the Ehlers transformation to the new $\mathcal{E}$
\begin{equation}
    \mathcal{E} \longrightarrow \mathcal{E}' = \frac{\mathcal{E}}{1+ic\mathcal{E}} = \underbrace{\frac{3r(2m-r)}{r^2(1-bc)^2+9c^2(2m-r)^2}}_{\displaystyle\Tilde{f}'} + i \underbrace{\frac{r^2b(1-bc)-9c(2m-r)^2}{r^2(1-bc)^2+9c^2(2m-r)^2}}_{\displaystyle h'}
\end{equation}
where obviously $\Omega\rightarrow\Omega$, $\gamma\rightarrow\gamma$, $\chi\rightarrow\chi$. Notice that the transformation adds a rotation term, indeed leads to the appearance of the field $\omega'$ that can be evaluated from $h'$ thanks to the definition (\ref{definizione_h_WLP})
\begin{equation}
    \omega' = -12 mc(bc-1) \cos\theta.
\end{equation}
Replacing the new field $f',\omega'$ in (\ref{WLP metric sferiche}) we obtain a modified BBMB solution 
\begin{equation}
    ds^2 = -\frac{(r-m)^2}{\eta}(dt+12mc(bc-1)\cos\theta d\varphi)^2 + \frac{\eta}{(r-m)^2}dr^2+ \eta(d\theta^2 + \sin^2\theta d\varphi^2),
\end{equation}
\begin{equation}
    \psi = \sqrt{\frac{6}{k}}\frac{m}{r-m} \notag
\end{equation}
where $\eta = r^2(1-bc)^2 +9c^2(2m-r)^2$. In order to recast this solution to a form that resemble the (\ref{BBMB_+_NUT_metric}), we can operate a change of variables 
\begin{align}
  r &\longrightarrow R = r\sqrt{\xi} - \frac{18c^2m}{\sqrt{\xi}} & t &\longrightarrow t' = \frac{t}{\sqrt{\xi}}
\end{align}
with $\xi = (1-bc)^2+9c^2$. Finally with the condition $b=(1\pm \sqrt{1-9c^2})/c$ and defining the parameters 
\begin{align}
  n &= \frac{6cm(1-bc)}{\sqrt{\xi}} & \mu &= \sqrt{m^2-n^2}
\end{align}
the metric becomes exactly solution (\ref{BBMB_+_NUT_metric}) with $n$ the NUT parameter.

\section{Charging BBMB black hole} \label{sezione bbmb + harrison}
Starting from the same change of variable (\ref{cambio cilindriche sferiche}) and the same identifications for $f_0$, $\omega_0$, $\omega$, $\chi$, $\gamma$ of the previous section, the Ernst's potentials reduces to $\mathcal{E}_0 = -3\left(1-\frac{2m}{r}\right)$ and $\boldsymbol{\Phi}_0=0$.
Then, the transformation V takes the form 
\begin{equation}
     \mathcal{E} \rightarrow \frac{\mathcal{E}_0}{1+3\Tilde{\alpha}^2\mathcal{E}_0-6\Tilde{\alpha}\boldsymbol{\Phi}_0} = \frac{-3(r-2m)}{r-9\Tilde{\alpha}^2(r-2m)} \notag
\end{equation}
\begin{equation}
    \boldsymbol{\Phi} \rightarrow \frac{\boldsymbol{\Phi}_0-\Tilde{\alpha}\mathcal{E}_0}{1+3\Tilde{\alpha}^2\mathcal{E}_0-6\Tilde{\alpha}\boldsymbol{\Phi}_0} = \frac{3\Tilde{\alpha}(r-2m)}{r-9\Tilde{\alpha}^2(r-2m)}. \notag
\end{equation}
From the definition of the Ernst's potential $\boldsymbol{\Phi} = A_t + i \Tilde{A}_{\varphi}$ we can see the appearance of the four potential $\bar{A} = A_t dt$ with $A_t = \boldsymbol{\Phi}$, since $\boldsymbol{\Phi}$ is real. At the same time we can evaluate $f$ inverting $\mathcal{E} = \Omega\chi f + 3 |\boldsymbol{\Phi}|^2 + i h$ with $h=0$, obtaining 
\begin{equation}
    f = \frac{(r-m)^2}{(r-9\Tilde{\alpha}^2(r-2m))^2}
\end{equation}
so that the charged solution is 
\begin{equation}
    ds^2=-\frac{(r-m)^2}{\eta^2}dt^2 + \frac{\eta^2}{(r-m)^2} dr^2 + \eta^2(d\theta^2 + \sin^2\theta d\varphi^2),
    \label{bbmb+ trasf. VI) sporca}
\end{equation}
\begin{align}
      \psi &=\sqrt{\frac{6}{k}}\frac{m}{r-m}, & \eta &= r-9\Tilde{\alpha}^2(r-2m), & \bar{A} &= \frac{3\Tilde{\alpha}(r-2m)}{\eta}dt. \notag
\end{align}
Again, we can propose a change of variable to recover the standard BBMB (\ref{BBMB metrics}), in particular 
\begin{align}
  r &\longrightarrow R = r(1-9\Tilde{\alpha}^2) + 18m\Tilde{\alpha}^2 & t &\longrightarrow t' = \frac{t}{1-9\Tilde{\alpha}^2}
\end{align}
so that redefining $m$ as $M = m (1+9\Tilde{\alpha}^2)$ and imposing 
\begin{equation}
    \Tilde{\alpha}^2 = \frac{1}{9}\frac{M - \sqrt{M^2-e^2}}{M + \sqrt{M^2-e^2}}
\end{equation}
the solution becomes
\begin{equation}
    ds^2=-\frac{(R-M)^2}{R^2}dt'^2 + \frac{R^2}{(R-M)^2} dR^2 + R^2(d\theta^2 + \sin^2\theta d\varphi^2),
    \label{bbmb + trasf V) in forma normale}
\end{equation}
\begin{align}
      \psi &=\sqrt{\frac{6}{k}}\frac{\sqrt{M^2-e^2}}{R-M}, & \bar{A} &= \left( -\frac{e}{R} + \sqrt{\frac{M - \sqrt{M^2-e^2}}{M + \sqrt{M^2-e^2}}}\right) dt' \notag
\end{align}
equivalent to the charged BBMB proposed in \citep{embedding}.

