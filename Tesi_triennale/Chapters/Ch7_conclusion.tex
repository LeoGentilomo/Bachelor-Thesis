\chapter{Conclusions}
\thispagestyle{empty}

\section{Conclusions}
In this thesis we have examined how a scalar-tensor theory significantly modifies the equations of motion of classical GR, as highlighted in Chapter 2. This led us to employ the Ernst's technique in the conformal frame, where the scalar field is naturally incorporated into the definition of $\mathcal{E}$, but not in the one of $\boldsymbol{\Phi}$. This suggests that the scalar field does not have significant couplings with the electromagnetic field.

We then explored the Lie point symmetries, observing that in the uncharged case, the number of symmetries triples, increasing from three to nine, with respect to GR. All symmetries present in the classical theory are also present in the conformal frame. At least four of the nine transformations are gauge symmetries. However, the other ones, specifically III, VI and VIII are clearly non-trivial, as they lead to solutions with different Petrov types.

The introduction of the electromagnetic field, as expected, did not produce any surprising new effects, as it involved the introduction of a gauge symmetry and transformation V (a conformal frame version of the Harrison transformation). This further confirmed that the scalar field does not have any particular coupling with the electromagnetic field.

Among the well-known transformations in classical GR, the most significant are those of Ehlers and Harrison, which allow respectively the addition of NUT/electric charge from a standard seed solution, or the immersion into a swirling/Melvin universe from a magnetic seed. As demonstrated in Chapter \ref{applications}, these properties are shared by transformations III and V in the conformal frame, then we can identify them as a conformal versions of the Ehlers and Harrison transformations.

Building on this identification, we constructed a new solution by immersing a Reissner–Nordström black hole with a scalar parameter into a swirling universe. This approach also revealed another interesting aspect: by turning off all energy densities in the solution, we find a solution that represents a scalar monople embedded in the swirling background. This leads us to propose an interesting future avenue, which involves taking a solution in the conformal frame embedded in Melvin universe and turning off the energy densities to observe how the background is modified. This operation, however, must be performed with a seed different from the BBMB solution, where turning off the mass causes the scalar field to vanish.

The transformation VI corresponds to the conformal version of the symmetry in the MC frame $\hat{\psi}\rightarrow\hat{\psi}+d$. In appendix \ref{appendice B} we confirm it is able to generate a traversable wormhole from the BBMB solution.

In conclusion, about the new non-trivial transformations founded, we cannot say much about them. In Chapter \ref{new solutions}, we derived two new solutions based on transformations VIII applied to the BBMB and RNS as seed solutions. However, beyond the classification according to their Petrov type, we have not yet conducted a thorough analysis of their physical implications. Our hope is that, in the future, these new transformations can be physical interpreted, as done for the Ehlers and Harrison transformations, and that they may offer insight about the solutions \ref{BBMB+VIII)} and \ref{R.N. PELOSO +VIII)}. Transformation IX and XI, which appears to be particularly complex, have not yet been analyzed due to the difficulty of them application and remains a subject for future investigation. In addition, we are uncertain about the transformation involving the coordinate $\rho$. The treatement through the symmetries of the effective action might be too naive. It would be better to further investigate those symmetries with the help of the prolongation technique.

In summary, we have laid the groundwork for a deeper understanding of symmetries and solutions in the conformal frame, but much work remains to be done, particularly in assigning a physical interpretation, if there is one, to the new transformations and solutions presented.






