\chapter{BBMB black hole in a swirling universe} \label{appendice C}
For the seek of completeness, in this appendix we show that even for magnetic seeds, transformation III behaves exactly as Ehlers transformation of classical GR. In particular we recover the BBMB black hole in the swirling background of section \ref{sub sezione swirling background}. 
\section{BBMB in swirling universe}
Let's report, for convenience, the BBMB solution (\ref{BBMB metrics}) 
\begin{equation}
    ds^2 = -\left(1-\frac{m}{r}\right)^2dt^2 + \left(1-\frac{m}{r}\right)^{-2}dr^2 + r^2(d\theta^2 + \sin^2\theta d\varphi^2),
    \notag
\end{equation}
\begin{equation}
    \psi = \sqrt{\frac{6}{k}}\frac{m}{r-m} \notag
\end{equation}
We can map the magnetic LWP (\ref{magnetic WLP}) in this solution by the extremal\footnote{Extremal because this change of coordinates it's the same used for RNS solution in the extremal case of $e^2+s = m^2$} change of coordinates (\ref{cambio cilindriche sferiche}) of section (\ref{sezione bbmb+nut}) and with the identifications 
\begin{equation*}
    \begin{aligned}
        \begin{cases}
           \displaystyle f_0  = -r^2\sin^2\theta \\
            \Omega = 1 \\
            \omega_0 = 0 \\
        \end{cases}
        \begin{cases}
           \displaystyle e^{2\gamma} = \frac{r^4\sin^\theta}{(r-m)^2} \\
            \displaystyle \chi = 3r\frac{2m-r}{(r-m)^2}. \\
        \end{cases}
    \end{aligned}
\end{equation*}
This times the Ernst's potentials are $\mathcal{E}_0=\alpha f_0=\Omega\chi f_0$ and $\boldsymbol{\Phi}_0=0$, so that transformation III becomes
\begin{equation}
    \mathcal{E}_0 \longrightarrow \mathcal{E} = \frac{\mathcal{E}_0}{1+ij\mathcal{E}_0} = \underbrace{\frac{-3r^3(2m-r)\sin^2\theta}{(r-m)^2\Lambda(r,\theta)}}_{\displaystyle \Tilde{f}} + i \underbrace{\frac{-j9r^6(2m-r)^2\sin^2\theta}{(r-m)^4\Lambda(r,\theta)}}_{\displaystyle h}
\end{equation}
with $\Lambda(r,\theta) = 1+j^2\mathcal{E}_0^2$ and $\boldsymbol{\Phi}=\boldsymbol{\Phi}_0=0$. Again Ehlers transformation adds a rotational term $\omega$ that we can evaluate from the definition of $h$ (\ref{definizione h magnetica}). In this coordinates system it becomes 
\begin{equation*}
    \begin{aligned}
        \begin{cases}
           \displaystyle\partial_r \omega = -\sin\theta \alpha^{-1}f^{-2}\partial_{\theta} h \\[0.7em]
           \displaystyle\partial_{\theta} \omega = (r-m)^2\sin\theta \alpha^{-1}f^{-2}\partial_{r} h .\\
        \end{cases}
    \end{aligned}
\end{equation*}
Combining everything, the BBMB solution in swirling universe is
\begin{equation}
    ds^2=\Lambda(r,\theta) \left[-\frac{(r-m)^2}{r^2}d\tau^2+ \frac{r^2}{(r-m)^2}dr^2+r^2d\theta^2 \right] + \frac{r^2\sin^2\theta}{\Lambda(r,\theta)}\left(d\phi -\omega d\tau \right)^2,
    \notag
\end{equation}
\begin{equation}
     \psi = \sqrt{\frac{6}{k}}\frac{m}{r-m}
        \label{BBMB magnetica + III)}
\end{equation}
where
\begin{align}
  \Lambda(r,\theta) & = 1+9j^2r^6\frac{(2m-r)^2}{(r-m)^4}\sin^4\theta, & \omega &= -\frac{12j}{r-m}(3m^2-3mr+r^2)\cos\theta .\notag
\end{align}
Notice that setting to zero the energy density ($m \rightarrow 0$) we recover the swirling background (\ref{swirling background}) instead of the conformal version (\ref{swirling background conforme}), because in the BBMB solution without the mass the scalar field vanishes.