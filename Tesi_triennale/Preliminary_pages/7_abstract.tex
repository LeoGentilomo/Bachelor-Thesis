
%%%%%%%%%%%%%%%%%%%%%%%%%%%%%%%%%%%%%%%%%%%%%%%%%%%%%%%%%%%%%%%%%%%%%%
%%%%%%%%%%%%%%%%%%%%%%%%%%% abstract %%%%%%%%%%%%%%%%%%%%%%%%%%%%%%%%%
%%%%%%%%%%%%%%%%%%%%%%%%%%%%%%%%%%%%%%%%%%%%%%%%%%%%%%%%%%%%%%%%%%%%%%


\addcontentsline{toc}{chapter}{Abstract}


\begin{center}
{\LARGE \textbf{Abstract}}
\end{center}


The abstract is a critical part of a scientific paper; in fact, it may be the only part people read. An abstract is a short summary of a longer work (such as a dissertation or thesis) usually about a paragraph ($\sim$350 words) long. The abstract concisely reports the aims and outcomes of your research so that readers know exactly what the thesis is about. Although the shortest section of a paper, writing an abstract if often considered the hardest section of a manuscript to write. Often limited by word length, writers must adequately and concisely summarize their research for broad audiences. The abstract should not exceed 500 words. Write the abstract last or revise you abstract, when you’ve completed the rest of the dissertation. For more details, refer guidelines for thesis preparation of IIT Ropar.\\

\textbf{Keywords}: Keywords 1; Keywords 2; Keywords 3; Keywords 4; Keywords 5; Keywords 6;

\newpage